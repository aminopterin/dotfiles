\begin{document}
\newcounter{cntSubSec}% counter used in subsections
\setcounter{cntSubSec}{0}% set counter to zero
\newcounter{cntNote}% counter used in subsections
\setcounter{cntNote}{0}% set counter to zero

\title{ \Huge{\It{浪狗行}} \large{\Tt{并序}} }
\author{}
\date{}
\maketitle
\vspace{-1.6cm}% removes vertical spacing 
\hfill{\textit{鄭子宇}, June 17, 2016}

\mySec{序}

\mySubSec{原文}

余每之\Cite{}電機二館、入門宿\Cite{}有浪狗二焉、一色黃、一色玄\Cite{}。輒勾留\Cite{}而歇、或吐舌以遣\Cite{}熱。偃伏而蜷、亦席地以寐晝。恬然自得、莫之能匹。噫、夫人者、豈弗如狗邪。今者食安殆\Cite{}焉、社福匱\Cite{}焉、內有政爭、外有敵伺。然後文士病\Cite{}之、咸稱鬼島。復觀彼其二浪狗者、猶仲尼時人長沮桀溺之疇\Cite{}也、而余輩孰得逸樂若此乎。讀書勤卷、思家國而愧然。慮政憂民、恐疲敝而惘然。則如之曷\Cite{}。故作浪狗行\Cite{}、托乎古意\Cite{}、聊以舒\Cite{}懷。

\mySubSec{注釋}
\setcounter{cntNote}{0}% set counter to zero
\Cite{}之\quad 到。
\Cite{}宿\quad 總是。
\Cite{}玄\quad 黑色。
\Cite{}勾留\quad 停留。
\Cite{}遣\quad 驅散。
\Cite{}殆\quad 危險。
\Cite{}匱\quad 缺乏。
\Cite{}病\quad 批評。
\Cite{}猶仲尼時人長沮桀溺之疇\quad 像是孔子當時的隱士,長沮和桀溺。事見《論語·微子》。疇,類別。
\Cite{}如之曷\quad 拿他怎麼辦,即如何。
\Cite{}行\quad 一種樂府的體裁。如白居易有《琵琶行》。
\Cite{}古意\quad 古代的風格。
\Cite{}舒\quad 抒發。

\mySubSec{語譯}

\mySec{詩}

\mySubSec{原文}

\mySubSec{注釋}

\mySubSec{語譯}

\end{document}

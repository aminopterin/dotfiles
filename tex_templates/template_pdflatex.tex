\documentclass[12pt]{article}

% To require the font used to be T1-encoding
\usepackage[T1]{fontenc}
% Set the default font to be Garamond.
% (Don't use `urw-garamond`, which is badly written.)
\usepackage{garamondx}
% For various support for math.
\usepackage[garamondx,cmbraces]{newtxmath}
% Alias for typewriter font: Latin Modern Typewriter
\renewcommand*\ttdefault{lmtt}
% Alias for sans-serif font: Helvetica
\renewcommand*\sfdefault{phv}

% % % % % % % % % % % % % % % % % % % % % % % % % % % % % % % % 

% For math typesetting.

% For general math formulae typography; a superset of `amsmath`.
\usepackage{mathtools}
% For several other math symbols.
\usepackage{amssymb}
% For graphics insertion; a superset of `graphics`.
\usepackage{graphicx}
% For colored text; a superset of `color`.
\usepackage{xcolor}

% To set fonts used for math.
% (Do not alter the order of this line, for strange reasons!)
\usepackage[
         frak=esstix, scr=boondoxo, cal=cm, bb=boondox
      ]{mathalfa}
% Available fonts as of each: (* has small cases)
% `mathfrak`: *`esstix`, *`boondox`, *`pxtx`
% `mathbb`: `ams`, `pazo`, `fourier`, `esstix`, *`boondox`, `px`, `txof`.
% `mathcal` & `mathscr`: `rsfs`, `rsfso`, `zapfc`, `pxtx`, *`esstix`, *`boondox`, *`boondoxo`, *`dutchcal`.

% % % % % % % % % % % % % % % % % % % % % % % % % % % % % % % % 

% Formatting:

% To set vertical spacing between adjacent paragraphs.
\setlength{\parskip}{1.5ex}
% To indent a paragraph always.
\setlength{\parindent}{4ex}
% To controls typesetting the title
\usepackage{titling}
% Decreases spacing followng the title
\setlength{\droptitle}{-2cm}
% To set page margins
\usepackage[
         top=2.1cm,
         bottom=1.9cm,
         left=1.8cm,
         right=1.8cm%
      ]{geometry}

% Adjustment of spacing of each section; see below.
\usepackage[compact]{titlesec}
% `\titlespacing` and `\titleformat` come from `titlesec`.
% Resp.: left margin, vertical spacing, seperation to text following.
\titlespacing{\section}{8ex}{*5}{*0}
% Same as above.
\titlespacing{\subsection}{2ex}{*1}{*2}
% Same as above.
\titlespacing{\subsubsection}{0pt}{*0}{*0}
% To modify the title of a subsection.
% No new-line before subsequent text
\titleformat{\subsection}[runin]{
         % color, font, shape
         \color{red}\large\bfseries\itshape
      }{}{}{}[]
% Subsection numbering.
\newcommand{\MySubSec}[1]{
         \addtocounter{CntSubSec}{1}
         \subsection*{\arabic{CntSubSec}. #1}
      }

% For verbatim quote of source code.
\usepackage{listings}
\definecolor{myMauve}{rgb}{0.58,0,0.82}
\lstset{
         numbers=left,
         frame=single,
         breaklines=true,
         basicstyle=\footnotesize\ttfamily\color{black},
         keywordstyle=\bfseries\color{teal},
         commentstyle=\itshape\color{gray},
         identifierstyle=\color{black},
         stringstyle=\color{blue}
      }

% To allow a table to break across a page break.
\usepackage{longtable}

% To increase row height of table with Chinese that's often too crowded.
\renewcommand{\arraystretch}{1.4}

% % % % % % % % % % % % % % % % % % % % % % % % % % % % % % % % 

% Greek alphabets:

% Lower case: (Only those with long name is aliased.)
\newcommand*\aG\alpha
\newcommand*\bG\beta
\newcommand*\gG\gamma
\newcommand*\dG\delta
\newcommand*\eG\varepsilon
\newcommand*\zG\zeta
\newcommand*\tG\vartheta
\newcommand*\kG\kappa
\newcommand*\lG\lambda
\newcommand*\sG\sigma
\newcommand*\fG\varphi
\newcommand*\oG\omega 

% Upper case: (Only those with long name is aliased.)
\newcommand*\GG\varGamma
\newcommand*\DG\varDelta
\newcommand*\TG\Theta
\newcommand*\LG\varLambda
\newcommand*\PG\varPi
\newcommand*\SG\varSigma
\newcommand*\FG\varPhi
\newcommand*\YG\varUpsilon
\newcommand*\OG\varOmega

% % % % % % % % % % % % % % % % % % % % % % % % % % % % % % % % 

% Other symbols:

\newcommand*\oo\infty% infinity, whose shape resembles "oo"
\newcommand*\F\frac% "F"raction
\newcommand*\R\sqrt% "R"oot
\newcommand*\M\cdot% "M"ultiply
\newcommand*\N\nabla% del sign
\newcommand*\X\times% cross, whose shape resembles "X"
\newcommand*\Pt\partial% "P"ar"T"ial differentiation
\newcommand*\V\mathbf% bold italic, e.g. "V"ectors
\newcommand*\Ev\forall% "Ev"ery
\newcommand*\Ex\exists% "Ex"ists
\newcommand*\Eq\Leftrightarrow% "Eq"ivelent
\newcommand*\Ip\Rightarrow% "I"m"p"lies
\newcommand*\Mp\mapsto% "M"a"p"
\newcommand*\Ctd\subseteq% "C"on"t"aine"d"
\newcommand*\Ctn\supseteq% "C"on"t"ai"n"ing
\newcommand*\ii{ \mathring{\imath} }% for imag. unit
\newcommand*\jj{ \mathring{\jmath} }% for imag. unit
\newcommand*\dd{ \BF{d} }% for differential
\newcommand*\ee{ \BF{e} }% for natural base

% % % % % % % % % % % % % % % % % % % % % % % % % % % % % % % % 

% Brackets, customized fonts:

\newcommand*{\Rb}[1]{ \left( #1 \right) }% "R"ound "b"racket, or commonly parenthesis
\newcommand*{\Sb}[1]{ \left[ #1 \right] }% ("S"quare) "b"racket
\newcommand*{\Cb}[1]{ \left\{ #1 \right\} }% ("C"urly) "b"race
\newcommand*{\Ab}[1]{ \left\langle #1 \right\rangle }% Chevrons, e.g. "A"ngle brackt
\newcommand*{\Nm}[1]{ \left| #1 \right| }% "N"or"m"
\newcommand*{\Bk}[2]{ \left\langle #1 \middle| #2 \right\rangle } % "B"ra-"K"et notation
% fonts for math. environment
\newcommand*{\BF}[1]{ \mathbb{#1} }% "B"lackboard "F"ont
\newcommand*{\CF}[1]{ \mathcal{#1} }% "C"ursive "F"ont
\newcommand*{\GF}[1]{ \mathfrak{#1} }% "G"othic "F"ont
\newcommand*{\SF}[1]{ \mathscr{#1} }% "S"cript "F"ont
\newcommand*{\Tw}[1]{\texttt{\bfseries{#1}}}% "t"ype"w"riter font
\newcommand*{\Ss}[1]{\textsf{#1}}% "S"ans-"s"erif
\newcommand*{\Emph}[1]{{\usefont{T1}{pbk}{b}{it} \color{blue}{#1}}}% my "empha"sis
\newcommand*{\Rm}[1]{\mathrm{#1}}% upright ("R"o"m"an) mode

% % % % % % % % % % % % % % % % % % % % % % % % % % % % % % % % 

% Miscellaneous:

\newcommand*{\It}{\textit}% Simply abbreviation.
\newcommand*{\Bf}{\textbf}% Simply abbreviation.
\newcommand*{\Df}{ \> \text{\raisebox{0.64pt}[0pt][0pt]{:}}\!\! = }% colon and equal
\renewcommand*{\l}{\label}% "L"a"b"el
\newcommand*{\Nt}{\notag\\}% "N"o"t"ag
\newcommand*{\Id}{\indent}% "I"n"d"ent
\newcommand*{\Tab}{\hspace{5em}}% "Tab", esp. in multi-line formulae
\newcommand*{\Q}{\hspace{1.5em}}% "Q"uadrat
% Just for generating dummy text.
\usepackage{lipsum}
% For hyperlink within the document.
\usepackage{hyperref}

% % % % % % % % % % % % % % % % % % % % % % % % % % % % % % % % 

% Equation environment:
% For strange reasons, when using these commands involving `\def`,
% one space must follow `{}`, otherwise parsing error results. 
\def\EqG #1 { \begin{gather} #1 \end{gather} }% Eqn. Gather
\def\EqGo #1 { \begin{gather*} #1 \end{gather*} }% unnumbered
\def\EqA #1 { \begin{align} #1 \end{align} }% Eqn. Align
\def\EqAo #1 { \begin{align*} #1 \end{align*} }% unnumbered
\def\Mtrx #1 { \begin{bmatrix} #1 \end{bmatrix} }% (bracketed) matrix

% % % % % % % % % % % % % % % % % % % % % % % % % % % % % % % % 

% Since `pdftex` is not suited for typeset of CJK, an article written
% mainly in CJK language shall be typeset with `xetex` or `luatex`.
% To Enable Chinese, Japanese, & Korean in UTF8 encoding.
\usepackage{CJKutf8}
\def\myCJK #1 { \begin{CJK}{UTF8}{bsmi} #1 \end{CJK} }% insertion of Chinese, Japanese, & Korean
